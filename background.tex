\chapter{Background}
\todo{more formal definitions}
First we must introduce the terminology and concepts of boolean logic including propositional logic, first order logic (including \gls{epr}) and quantified propositional logic.
After the terminology has been introduced the complexity classes of the satisfiability problem of each logic will be discussed.
Finally the concept of automated reasoning of both first order logic and \gls{qbf} will be detailed.

\section{Boolean Logic and Satisfiability}
Boolean logics and \gls{sat} form a calculus which can be used to reason about statements. There are different formal systems of logic that can be used for different purposes, we shall look at propositional logic, \gls{qbf} and first order logic.

\subsection{Propositional Logic}
A propositional variable $p$ can take one of two values; either $true$ or $false$. A variable can be negated using the \gls{not} symbol which reverses its value. If $p$ was $true$ then $\neg p$ is $false$ and vice versa. We will call a variable or its negation a \textit{literal} and denote the positive literal by $l$ and the negative literal by $\bar{l}$.
Boolean formulas are constructed from propositional variables built using the logical connectives \gls{or}, \gls{and} and \gls{implies}. A typical boolean formula might look like $(x \land y) \to z$. A more formal definition can be found in appendix~\ref{formaldefinitions}.

The satisfiability of a boolean formula is a decision problem that asks if an assignment of truth values to propositional variables can make the boolean formula true. The aforementioned logical connectives tell us how to combine the truth values of two variables. With a disjunction, $x \lor y$ is true if either of $x$ or $y$ or both are true. In a conjunction, $x \land y$ is true if both $x$ and $y$ are true. An implication $x \to y$ can be read as ``if $x$ is true then $y$ is true.'' with the case of $x$ being $false$ defined as being vacuously true. In the previous example of $(x \land y) \to z$ we can see that it is satisfiable as the assignment $x := true; y := true; z := true$ makes it true.

We require a more standard form of boolean formula that is easier to describe as an input format. For this we will use \gls{cnf}. \Gls{cnf} is a conjunction of clauses and a clause is a disjunction of literals. For example, a clause might be $(p \lor \neg q)$ and a full formula in \gls{cnf} might look like $(p \lor r) \land (\neg r \lor q) \land (q)$. Using \gls{cnf} allows us to more easily work with boolean formulas algorithmically.

\subsection{Quantified Boolean Formulas} \label{qbf}
\Gls{qbf} extends propositional logic with the \gls{forall} and \gls{exists} quantifiers. The statement $\forall x \exists y (x \lor y)$ states that for every assignment of $x$ there is at least one assignment of $y$ such that the formula $(x \lor y)$ is true. We can see that this \gls{qbf} is satisfiable. If $x := true$ then the formula is true and any assignment of $y$ will work. If $x := false$ then the assignment $y := false$ makes the formula false but the assignment $y := true$ does make the formula true. Therefore, for any assignment of $x$ there exists an assignment of $y$ such that the formula is true. A more formal definition can be found in appendix~\ref{formaldefinitions}.

In the most general case a quantified variable can appear in front of any sub-formula of a \gls{qbf}. Again we need a more standard form of \gls{qbf} that we can deal with algorithmically. This form is called \textit{prenex} \gls{cnf} (however we may refer to it by just \gls{cnf} assuming that the formula is prenexed). Any \gls{qbf} is logically equivalent to a \gls{cnf} formula and the process for transforming the \gls{qbf} into \gls{cnf} is called \textit{prenexing}. This process uses rewriting rules to move all the quantifiers in the formula to the leftmost part of the formula resulting in a \textit{quantifier prefix}. For example, $(\neg (\exists x A) \land B)$ is equivalent to $\forall x (\neg A \land B)$. Because all \glspl{qbf} are equivalent to some \gls{qbf} in \gls{cnf} we shall assume that any \gls{qbf} we are dealing with is already in \gls{cnf}.

Another useful notion will be the idea of an order to a prenexed \gls{qbf}'s prefix. If a variable $x$ is quantified to the left of another variable $y$ in the prefix such as $\exists x \forall y$ we say that $x$ is quantified before $y$. Using this idea we can assign a quantification level to the variables so that $x$ has a quantification level of 1 and $y$ has a quantification level of 2. The variable with the lowest quantification level is called the outermost quantified variable and the variable with the highest quantification level is the innermost.

\subsection{First Order Logic}
First order logic uses propositions that take variables or functions as arguments to form its formulas. These variables range over a specified problem domain such as the natural numbers. For example in the domain of the natural numbers the formula $\forall n \exists m P(n, m)$ where $P(n, m) = m > n$ is true; for any natural number $n$ there is a number $m$ that is larger than $n$. This differs from our previous definition of \gls{qbf} in that \gls{qbf} deals with only variables in a two valued domain (i.e. boolean) and does not have propositions. A more formal definition can be found in appendix~\ref{formaldefinitions}.

Our notions of prenexed \gls{cnf} also extend to first order logic.

As with propositional logic and \gls{qbf} we require a way to write our formulas that is convenient to work with. In this case we will use \gls{epr}, formally known as the \textit{Bernays-Sch{\"o}nfinkel} class of formulas. A formula is in \gls{epr} form if when it is written in \gls{cnf} it has the quantifier prefix $\exists * \forall *$ and contains no functions. This format will be useful because we can solve these problems using first order logic theorem provers.

\section{Complexity of Satisfiability}
Boolean logics are significant in complexity theory as they are standard embeddings for other problems in their complexity classes.

\subsection{SAT is NP-complete}
An NP problem is one that can be solved by a non-deterministic algorithm that runs in time relative to a polynomial of the size of the input. \Gls{sat} is one such problem. Stephen Cook proved in 1971 that the \gls{sat} problem is NP-complete~\cite{cook1971complexity} meaning that any other NP problem can be reduced to the \gls{sat} problem. This spurs lots of interest into \gls{sat} solvers because if we can solve the \gls{sat} problem in polynomial time then we can solve any NP problem in polynomial time too. This is known as the P=NP problem.

\subsection{QBF is PSPACE-complete}
A PSPACE problem is one that can be solved by a deterministic algorithm that runs using space in memory relative to a polynomial of the size of the input. \Gls{qbf} belongs to this complexity class and can be shown to be PSPACE-complete using Savitch's theorem~\cite{savitch1970relationships}. NP problems are a subset of PSPACE problems and it is not yet known if the two classes are equal or not. Because these algorithms run much slower than general \gls{sat} algorithms there has been much less interest in \gls{qbf} solvers outside academia compared to \gls{sat} solvers.

\subsection{EPR is NEXPTIME-complete} \label{eprnexptime}
Similarly to NP problems, NEXPTIME problems are solved by non deterministic algorithms running in time relative to an exponential of the size of the input. \Gls{epr} was shown to be NEXPTIME-complete by Harry Lewis in 1980~\cite{lewis1980complexity}. While this does mean that algorithms for proving satisfiability of \gls{epr} are in general slower than algorithms for propositional satisfiability we are still interested to see how \gls{epr} solvers fare when given inputs derived from \glspl{qbf}.

\section{Automated Reasoning}
Automated reasoning tools use algorithms to solve these \gls{sat} problems as well as other more general logic based problems based around deduction to find a result that isn't necessarily satisfiability. This brings artificial intelligence interests into the research in an effort to create artificial intelligences that can perform deductive reasoning. This project makes use of the automated reasoning tool iProver~\cite{korovin2008iprover} to solve the \gls{sat} problems generated by \textit{qbftoepr}.
