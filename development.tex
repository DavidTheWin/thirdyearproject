\chapter{Development}
\todo[inline]{Development goes here}
This chapter will describe the design and implementation of \textit{qbftoepr} at a high level with some of the technical choices and why they were made as well as some of the optimizations and compromises that were taken to improve performance.

\section{Language Choice}
The project was implemented in the language OCaml~\cite{ocaml} which is a functional language notable for the language extensions that give it object oriented and imperative functionality. The main motivator for implementing \textit{qbftoepr} in OCaml is the fact that iProver is implemented in OCaml which provides three advantages:

\begin{itemize}
\item Code from iProver can be re-used easily in \textit{qbftoepr}\\
\item Using the same language in \textit{qbftoepr} as iProver makes post-project maintenance easier\\
\item Using the same language would allow tighter integration when \textit{qbftoepr} passes its output to iProver (though this wasn't done in practice)\\
\end{itemize}

Even without these advantages OCaml would be a suitable language given that it has been designed with performance as a priority and has excellent built-in functions for manipulating lists of which the project makes extensive use.

\section{Input and Output Formats}
The input and output formats were an important choice but one that was relatively easy to make. The decision was to go with the standards already in use by other theorem provers to be able to compare \textit{qbftoepr} to them on exactly the same inputs. Another consideration was that the input format had to have a relatively simple grammar to simplify the input handling and the output format had to be simple enough to make printing the output efficient. The output format must also be accepted as an input format by iProver.

\subsection{QDIMACS}
QDIMACS~\cite{qdimacs} is the input format decided on for \textit{qbftoepr}. It is the format used by the QBFLIB~\cite{qbflib} which is a library of \gls{qbf} problem instances. QBFLIB also holds a competition called QBFEVAL which tests \gls{qbf} solvers against each other using the QDIMACS format. This makes it the standard of the \gls{qbf} research community and a clear choice of input format for \textit{qbftoepr}. It also opens access to the QBFLIB problem library providing test cases of all difficulty scales and the results of other solvers on these test cases. It also has a very simple grammar meaning that parsing a QDIMACS input is simple.

Below is our example from chapter~\ref{chapter:2} in QDIMACS form.
\begin{lstlisting}
c this is a comment
p cnf 4 3
a 1 2 3 0
e 4 0
3 4 0
2 -4 0
-2 1 0
\end{lstlisting}
Lines beginning with \textit{c} are comments, \textit{p} denotes the problem line which tells us the problem is in \gls{cnf} and has four variables and three clauses. Lines prefixed with \textit{a} are universally quantified variables, \textit{e} are existential. Lines after the prefix form the matrix which is the list of clauses. Numbers represent variables and a negative number represents the negative literal of that variable, i.e. $\bar{l}$. The prefix and matrix lines are appended with a zero as a line terminator.

\subsection{TPTP}
The output format chosen for \textit{qbftoepr} is TPTP~\cite{tptp}. Similarly to QBFLIB, the TPTP is a large problem library for automated theorem proving and as such it is one of the input formats that iProver accepts. The popularity of the TPTP library means it is also used by other automated theorem provers. Because the TPTP is a general problem library (whereas QBFLIB is specifically \gls{qbf} problems) its grammar is very complicated. However \textit{qbftoepr} does not need to parse it to output it and only requires a small subset of the format's features to output the result of the \gls{epr} conversion process.

Below is the example from chapter~\ref{chapter:2} after being converted to \gls{epr} in the TPTP output format.
\begin{lstlisting}
cnf(cl_0,plain,(p(X3) | p_f_4(X1,X2,X3))).
cnf(cl_1,plain,(p(X2) | ~p_f_4(X1,X2,X3))).
cnf(cl_2,plain,(~p(X2) | p(X1))).
cnf(cl_3,plain,(p(true))).
cnf(cl_4,plain,(~p(false))).
\end{lstlisting}
Each line is a clause, named \textit{cl\_x} where x is an identifier. Following is the \textit{plain} keyword which says there are no user defined semantics then the list of literals where \textit{\textasciitilde} denotes the negation of a literal.

\section{Data Structures}

\section{Implementation}
\subsection{Skolemization}
\subsection{Removing Functions}
\subsection{Dependency Scheme Construction}
\subsection{Complexity}
% raising to fol is linear in the number of clauses (and in their length?)? just translate symbols one at a time plus two extra clauses but that's dominated by the linearity of the symbol translation
% skolemization: finding existential variables linear in the length of the prefix n, then one linear search of the matrix for each variable found, another n*m? (n*m)^2? disclaimer this might be wrong
% removing functions is linear in the number of clauses (and in their length?)? replacing functions with predicates one by one

\section{Optimizations}

\section{Future Work}
\subsection{Dependency Scheme Optimizations}
\subsection{Anti-prenexing}
