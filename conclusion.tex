\chapter{Conclusion}
Finally, this chapter will summarise the project first looking at the progress then some closing analysis.

\section{Progress}
The project progressed slowly at first. The mix of multiple new technologies to learn at once delayed the start of the project until it was roughly six weeks behind. However from the implementation of Skolemization the project proceeded according to plan until the project demo. At this stage the project was extremely inefficient and not in a suitable state for providing evaluation results for the project demo. This meant that time had to be spent on improving the basic functionality rather than implementing some of the extensions that were originally planned beyond just the standard dependency scheme. The gantt chart in appendix~\ref{ganttchart} shows the planned time for each stage of the development.

\section{Conclusion}
The aim of the project was to produce a tool that would take a \gls{qbf} as input and produce an \gls{epr} output to be used in iProver. This was completed in the estimated time frame with extensions to the original functionality to use the standard dependency scheme over just the trivial dependency scheme. Some extensions such as the triangle dependency scheme and anti-prenexing discussed in chapter~\ref{futurework} were not implemented due to time constraints. It does not perform as well as competing tools but with more time the implementation could be improved to the point where it would be faster than the competing tools.

Overall the project was a valuable learning experience from the first contact with a functional language to researching beyond the undergraduate course to understand the background of the project.
